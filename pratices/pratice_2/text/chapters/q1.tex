\chapter{Questao 1}

\subsection*{Cenário de race condition}

Para identificação do cenário, foi desenvolvido o código abaixo que cria duas threads que iteram 10 vezes chamando as funções `aloca\_recurso' e `desaloca\_recurso'.

\begin{lstlisting}[style=CStyle]
    #include <stdio.h>
    #include <unistd.h>
    #include <pthread.h>
    #include <stdlib.h>
    #include <sys/time.h>
    
    #define MAX 10;
    int RECURSO_MAXIMO = MAX;
    
    // Cada thread utiliza as mesmas funções
    
    // Essa função aloca a quantidade de recursos desejada, se existir
    int aloca_recurso (int recurso_desejado) {
        if (RECURSO_MAXIMO - recurso_desejado < 0)
            return 0;
        else {
            RECURSO_MAXIMO -= recurso_desejado;
            return 0;
        }
    }
    
    // Essa função devolve uma certa quantidade de recursos alocados
    int desaloca_recurso(int recurso_devolvido) {
        if (RECURSO_MAXIMO + recurso_devolvido > 10)
            return 0;
        
        RECURSO_MAXIMO += recurso_devolvido;
        return 0;
    }
    
    
    void *thread1() {
        printf("Olá, sou a thread 1!\n");
        int i = 0;
        // Começo da sessão crítica
        while (i < 10) {
        aloca_recurso(i);
        sleep((rand() % 10 + 1)/10);
        desaloca_recurso(i);
        printf("t1: RECURSO_MAXIMO: %d\n", RECURSO_MAXIMO);
        i++;
        }
        // Fim da sessão crítica
        pthread_exit(NULL);
    }
    
    void *thread2() {
        printf("Olá, sou a thread 2!\n");
        int i = 0;
        // Começo da sessão crítica
        while (i < 10) {
        aloca_recurso(i);
        sleep((rand() % 10+1)/10);
        desaloca_recurso(i);
        printf("t2: RECURSO_MAXIMO: %d\n", RECURSO_MAXIMO);
        }
        // Fim da sessão crítica
        pthread_exit(NULL);
    }
    
    int main() {
        pthread_t thr[2];
    
        int i = 0;
        if(pthread_create(&thr[i], NULL, thread1, NULL)) {
            printf("Ops... Houve um erro na criação da thread %d.\n", i);
            return 0;
        }
        i++;
    
        if(pthread_create(&thr[i], NULL, thread2, NULL)) {
            printf("Ops... Houve um erro na criação da thread %d.\n", i);
            return 0;
        }
        printf("Final: RECURSO_MAXIMO: %d\n", RECURSO_MAXIMO);
    }
\end{lstlisting}

Explicação

Durante cada iteração, é printado o valor da variável RECURSO\_MAXIMO e a thread em questão. 
Verificou-se que cada vez o programa retornava valores diferentes em ordens diferentes para a variável RECURSO\_MAXIMO. Portanto, ocorre o race condition durante a execução do código.
O race condition ocorre a partir do momento em que as threads entram em suas respectivas zonas críticas (explicitadas no código), onde, ao entrar nos respectivios loops, ocorro a concorrência entre elas.

\subsection*{Soluçã Proposta}
A solução proposta utiliza o protocolo test-and-set.

Código

\begin{lstlisting}[style=CStyle]
#include <stdio.h>
#include <unistd.h>
#include <pthread.h>
#include <stdlib.h>
#include <sys/time.h>

#define MAX 10;
int RECURSO_MAXIMO = MAX;

typedef struct __lock_t {int flag;} lock_t;

void init(lock_t *mutex) {
    mutex->flag = 0;
}

void lock(lock_t *mutex) {
    while (mutex->flag == 1); //spin-wait
    mutex->flag = 1;
}

void unlock(lock_t *mutex) {
    mutex->flag = 0;
}

// Cada thread utiliza as mesmas funções

// Essa função aloca a quantidade de recursos desejada, se existir
int aloca_recurso (int recurso_desejado) {
    if (RECURSO_MAXIMO - recurso_desejado < 0)
        return 0;
    else {
        RECURSO_MAXIMO -= recurso_desejado;
        return 0;
    }
}

// Essa função devolve uma certa quantidade de recursos alocados
int desaloca_recurso(int recurso_devolvido) {
    if (RECURSO_MAXIMO + recurso_devolvido > 10)
        return 0;
    
    RECURSO_MAXIMO += recurso_devolvido;
    return 0;
}


void *thread1() {
    printf("Olá, sou a thread 1!\n");
    int i = 0;
    // Começo da sessão crítica
    lock_t *MUTEX;
    init(MUTEX);
    lock(MUTEX);
    while (i < 10) {
    aloca_recurso(i);
    sleep((rand() % 10 + 1)/10);
    desaloca_recurso(i);
    printf("t1: RECURSO_MAXIMO: %d\n", RECURSO_MAXIMO);
    i++;
    }
    unlock(MUTEX);
   // Fim da sessão crítica
    pthread_exit(NULL);
}

void *thread2() {
    printf("Olá, sou a thread 2!\n");
    int i = 0;
    // Começo da sessão crítica
    lock_t *MUTEX;
    init(MUTEX);
    lock(MUTEX);    
    while (i < 10) {
    aloca_recurso(i);
    sleep((rand() % 10+1)/10);
    desaloca_recurso(i);
    printf("t2: RECURSO_MAXIMO: %d\n", RECURSO_MAXIMO);
    }
    unlock(MUTEX);
    // Fim da sessão crítica
    pthread_exit(NULL);
}

int main() {
    pthread_t thr[2];

    int i = 0;
    if(pthread_create(&thr[i], NULL, thread1, NULL)) {
        printf("Ops... Houve um erro na criação da thread %d.\n", i);
        return 0;
    }
    i++;

    if(pthread_create(&thr[i], NULL, thread2, NULL)) {
        printf("Ops... Houve um erro na criação da thread %d.\n", i);
        return 0;
    }
    printf("Final: RECURSO_MAXIMO: %d\n", RECURSO_MAXIMO);
}
\end{lstlisting}
