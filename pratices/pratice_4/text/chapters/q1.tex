\chapter{Questao 1}

\subsection*{Parte A}

Na situação proposta, temos os seguintes pontos:
- O SO alocou 1 página para o texto do código
- O SO alocou 3 páginas para as matrizes
- Cada matriz ocupa 16 páginas de memória
- A operação da linha 7 é executada 4096 (64 x 64) vezes
- Para cada execução da linha 7, é executada
 uma operação de leitura nas matrizes A e B,
 e uma operação de escrita na matriz C

Para o cálculo, considera-se que o SO em questão utiliza a
 gestão de page replacement LRU (Least Recently Used). Com isso,
 pode-se desconsiderar a página alocada para o texto do código,
 pois ela nunca será a página menos recentemente usada.

Temos as seguintes variáveis:
 - Número de execuções da linha 7: 4096
 - Quantidade de páginas alocadas*: 3
 - Quantidade de páginas por matriz: 16
 - Quantidade de matrizes: 3

*considerando apenas as páginas das matrizes

Dado a forma de posicionamento descrita no enunciado, e que o 
acesso às matrizes é sequencial. Temos que a cada 3 páginas de
 execução, haverá 1 page fault para cada matriz, ou seja, 3 page
 faults ao todo.

Então temos que:
 - Quantidade de page faults a cada 3 páginas de execução: 3


Com essas informações, é possível fazer o cálculo da quantidade de
 page faults na execução do programa. Segue a implementação abaixo.

 \begin{lstlisting}[style=CStyle]
    #include <stdio.h>
    #define Size 64
    
    int main() {
        int A[Size][Size], B[Size][Size], C[Size][Size];
        int i, j;
        int page_faults = -3; // variável para contagem dos page_faults
        for (j = 0; j< Size; j ++) {
            for (i = 0; i< Size; i++) {
                C[i][j] = A[i][j] + B[i][j];
                if (i % 3 == 0) page_faults += 3; // A cada 3 iterações, acontecem 3 page_faults
            }
        }
    
        printf("page_faults: %d\n", page_faults);
    
        return 0;
    }
 \end{lstlisting}
 
Obs: a variável `page_faults` é inicializada em -3 para
 desconsiderar a primeira iteração, pois na primeira iteração
 não ocorre o page fault. 

Com a execução desse programa, obteve-se o total de \textbf{4221}
 page faults. A frequência de page faults para cada iteração seria então:

 \[4221/4096 = 1.03 page faults por iteração\]



\subsection*{Parte B}

O grande problema que causa o número elevado de page faults é que,
 devido ao posicionamento da matriz nas páginas de memória, o
 programa não utiliza todos os valores da matriz presentes naquela
 página. Tornando-se necessário acessar aquela página novamente
 em uma outra iteração.

A mudança para resolver isso seria trocar os iteradores `i` e `j` de lugar.
 Com essa mudança, todos os valores de uma dada página seriam acessados todos de
 no mesmo acesso àquela página. E com isso, não seria acessar a mesma
 página várias vezes.

Segue a implementação dessa mudança:

\begin{lstlisting}[style=CStyle]
    #include <stdio.h>
    #define Size 64
    
    int main() {
        int A[Size][Size], B[Size][Size], C[Size][Size];
        int i, j;
        int page_faults = -3; // variável para contagem dos page_faults
        for (j = 0; j< Size; j ++) {
            for (i = 0; i< Size; i++) {
                C[j][i] = A[j][i] + B[j][i];
            }
            if (j % 3 == 0) page_faults += 3; // A cada 3 iterações, acontecem 3 page_faults
        }
    
        printf("page_faults: %d\n", page_faults);
    
        return 0;
    }
\end{lstlisting}

Com essa mudança, o número total de page
 faults caiu para \textbf{63}