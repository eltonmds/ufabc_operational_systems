\chapter{Questao 1}

A solução proposta utiliza o método `named pipe` onde há um processo responsável por criar um arquivo através da função fifo, o processo coletor e o displayer se comunicam através desse arquivo.
Segue abaixo o código em C.

\subsection*{named\_pipe}
Processo responsável por criar o arquivo utlizando a função fifo.

\begin{lstlisting}[style=CStyle]
    #include <stdio.h>
    #include <stdlib.h>
    #include <sys/stat.h>
    #include <sys/types.h>
    
    
    int main() {
        const char* pipeF = "fifoF";
    
        if (mkfifo(pipeF, 0777) < 0) {
            printf("Ops...erro:");
            exit(1);
        }
        else {
            printf("Queue named pipe feita!!!\n");
        }
    }
\end{lstlisting}


\subsection*{coletor}

O processo coletor é responsável por escrever as cotações no arquivo fifo
\begin{lstlisting}[style=CStyle]
    #include <stdio.h>
    #include <stdlib.h>
    #include <fcntl.h>
    #include <sys/stat.h>
    #include <sys/types.h>
    #include <unistd.h>
    #include <stdlib.h>

    #define FIFO_FILE "fifoF"

    struct cotacao {
        char moeda_origem[4];
        char moeda_destino[4];
        float valor;
        char data_hora[20];
    };

    int main() {
        int fd;
        struct cotacao cotacoes[15] = {
            {"USD", "BRL", 5.25, "2023-01-20T10:31:00"},
            {"EUR", "BRL", 6.20, "2023-01-20T10:31:00"},
            {"CNY", "BRL", 0.76, "2023-01-20T10:31:00"},
            {"RUB", "BRL", 0.07, "2023-01-20T10:31:00"},
            {"ARS", "BRL", 0.03, "2023-01-20T10:31:00"},
            {"ASA", "BRL", 0.30, "2023-01-20T10:31:00"},
            {"ABL", "BRL", 0.35, "2023-01-20T10:31:00"},
            {"SFR", "BRL", 5.35, "2023-01-20T10:31:00"},
            {"LEV", "BRL", 1.35, "2023-01-20T10:31:00"},
            {"BAR", "BRL", 3.35, "2023-01-20T10:31:00"},
            {"THS", "BRL", 2.35, "2023-01-20T10:31:00"},
            {"BTC", "BRL", 145000.25, "2023-01-20T10:31:00"},
            {"ETH", "BRL", 9430.35, "2023-01-20T10:31:00"},
            {"ADA", "BRL", 1.55, "2023-01-20T10:31:00"},
            {"AJA", "BRL", 2.35, "2023-01-20T10:31:00"},
            {"BOC", "BRL", 0.75, "2023-01-20T10:31:00"},
        };

        fd = open(FIFO_FILE, O_WRONLY);

        if (fd < 0) {
            perror("open");
            exit(1);
        }

        for (int i = 0; i < 15; i++) {
            write(fd, &cotacoes[i], sizeof(struct cotacao));
        }

        close(fd);

        return 0;
    }
\end{lstlisting}

\subsection*{displayer}

o processo displayer lê as informçaões escritas pelo coletor no arquivo fifo e apresenta as informações.

\begin{lstlisting}[style=CStyle]
    #include <stdio.h>
    #include <fcntl.h>
    #include <sys/stat.h>
    #include <sys/types.h>
    #include <unistd.h>
    
    #define FIFO_FILE "fifoF"
    
    struct cotacao {
        char moeda_origem[4];
        char moeda_destino[4];
        float valor;
        char data_hora[20];
    };
    
    int main() {
        int fd;
        struct cotacao cotacao;
    
        fd = open(FIFO_FILE, O_RDONLY);
    
        while (read(fd, &cotacao, sizeof(struct cotacao)) > 0) {
            printf("Moeda origem: %s\n", cotacao.moeda_origem);
            printf("Moeda destino: %s\n", cotacao.moeda_destino);
            printf("Valor: %.2f\n", cotacao.valor);
            printf("Data-hora: %s\n", cotacao.data_hora);
            printf("\n");
        }
    
        close(fd);
    
        return 0;
    }
    
\end{lstlisting}