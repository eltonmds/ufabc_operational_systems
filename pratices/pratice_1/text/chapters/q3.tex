\graphicspath{ {./images/} }

\chapter{Questão 3}

\subsection*{Código}

Versão com thread

\begin{lstlisting}[style=CStyle]
    #include <stdio.h>
    #include <unistd.h>
    #include <pthread.h>
    #include <stdlib.h>
    #include <sys/time.h>
    
    
    int vector[20];
    int min, max, mode;
    float avg;
    
    void *maxVector() {
    
        int iterator;
        max = vector[0];
        for (iterator = 1; iterator < 19; iterator++) {
           if (vector[iterator] > max) max = vector[iterator];
        }
        pthread_exit(NULL);
    }
    
    void *minVector() {
    
        int iterator;
        min = vector[0];
        for (iterator = 1; iterator < 19; iterator++) {
            if (vector[iterator] < min) min = vector[iterator];
        }
        pthread_exit(NULL);
    }
    
    void *avgVector() {
    
        int iterator;
        int ans = 0;
        for (iterator = 0; iterator < 19; iterator++) {
            ans += vector[iterator];
        }
    
        avg =  ans/20;
        pthread_exit(NULL);
    }
    
    void *modeVector() {
       
       int iterator;   
       int maxCount = 0, j;
       mode = 0;
       for (iterator = 0; iterator < 20; ++iterator) {
          int count = 0;
          
          for (j = 0; j < 20; ++j) {
             if (vector[j] == vector[iterator])
             ++count;
          }
          
          if (count > maxCount) {
             maxCount = count;
             mode = vector[iterator];
          }
       }
    }
    
    int main() {
        struct timeval current_time;
        
        gettimeofday(&current_time, NULL);
        printf("seconds : \%ld\nmicro seconds : %ld",    
        current_time.tv_sec, current_time.tv_usec
        );
        printf("\n");
    
    
        pthread_t thr[4];
    
        for (int i = 0; i < 20; i++) {
            vector[i] = rand() \% 10 + 1;
        }
    
        int i = 0;
        if(pthread_create(&thr[i], NULL, maxVector, NULL)) {
            printf("Ops... Houve um erro na criação da thread \%d.\n", i);
            return 0;
        }
        i++;
        if(pthread_create(&thr[i], NULL, minVector, NULL)) {
            printf("Ops... Houve um erro na criação da thread \%d.\n", i);
            return 0;
        }
        i++;
        if(pthread_create(&thr[i], NULL, avgVector, NULL)) {
            printf("Ops... Houve um erro na criação da thread \%d.\n", i);
            return 0;
        }
        i++;
        if(pthread_create(&thr[i], NULL, modeVector, NULL)) {
            printf("Ops... Houve um erro na criação da thread \%d.\n", i);
            return 0;
        }
    
        (void) pthread_join(thr[0], NULL);
        (void) pthread_join(thr[1], NULL);
        (void) pthread_join(thr[2], NULL);
        (void) pthread_join(thr[3], NULL);
    
        printf("O valor do vetor é:\n");
        for (int i = 0; i < 20; i++) {
            printf("\%d ", vector[i]);
        }
        printf("\n");
    
        printf("O valor máximo é: \%d.\n", max);
        printf("O valor mínimo é: \%d.\n", min);
        printf("O valor da moda é: \%d.\n", mode);
        printf("O valor da média é: \%.2f.\n", avg);
        
        gettimeofday(&current_time, NULL);
        printf("seconds : %ld\nmicro seconds : \%ld",    
        current_time.tv_sec, current_time.tv_usec
        );
        printf("\n");
        pthread_exit(NULL);
    }    
\end{lstlisting}

Versão sem thread

\begin{lstlisting}[style=CStyle]
    #include <stdio.h>
    #include <stdlib.h>
    #include <sys/time.h>
    
    int vector[20];
    int min, max, mode;
    float avg;
    
    void maxVector() {
        int iterator;
        max = vector[0];
        for (iterator = 1; iterator < 19; iterator++) {
            if (vector[iterator] > max) max = vector[iterator];
        }
    }
    
    void minVector() {
        int iterator;
        min = vector[0];
        for (iterator = 1; iterator < 19; iterator++) {
            if (vector[iterator] < min) min = vector[iterator];
        }
    }
    
    void avgVector() {
        int iterator;
        int ans = 0;
        for (iterator = 0; iterator < 19; iterator++) {
            ans += vector[iterator];
        }
    
        avg =  ans/20;
        
    }
    
    void modeVector() {
       int iterator;   
       int maxCount = 0, j;
       mode = 0;
       for (iterator = 0; iterator < 20; ++iterator) {
          int count = 0;
          
          for (j = 0; j < 20; ++j) {
             if (vector[j] == vector[iterator])
             ++count;
          }
          
          if (count > maxCount) {
             maxCount = count;
             mode = vector[iterator];
          }
       }
    
    }
    
    int main() {
        struct timeval current_time;
        
        gettimeofday(&current_time, NULL);
        printf("seconds : \%ld\nmicro seconds : \%ld",    
        current_time.tv_sec, current_time.tv_usec
        );
        printf("\n");
    
        for (int i = 0; i < 20; i++) {
            vector[i] = rand() \% 10 + 1;
        }
    
        maxVector();
        minVector();
        avgVector();
        modeVector();
    
        printf("O valor do vetor é:\n");
        for (int i = 0; i < 20; i++) {
            printf("\%d ", vector[i]);
        }
        printf("\n");
    
        printf("O valor máximo é: \%d.\n", max);
        printf("O valor mínimo é: \%d.\n", min);
        printf("O valor da moda é: \%d.\n", mode);
        printf("O valor da média é: \%.2f.\n", avg);
        
        gettimeofday(&current_time, NULL);
        printf("seconds : \%ld\nmicro seconds : \%ld",    
        current_time.tv_sec, current_time.tv_usec
        );
        printf("\n");
    
        return 0;
    }    
\end{lstlisting}

\subsection*{Explicação dos Resultados}
Processador da máquina: 11th Gen Intel® Core™ i7-11390H @ 3.40GHz, 4 núcleos, 8 threads
O programa sem thread obteve um resultado muito melhor. cerca de 30 vezes mais rápido.
Isso se deve ao fato de que, devido ao tamanho do vetor ser muito pequeno, é mais custoso criar as threads e criar um paralelismo do que rodar mono thread.
Para comprovar isso, foi feito um novo teste com um vetor maior, com 20000 itens. Desta vez, o programa com thread obteve um resultado melhor.
