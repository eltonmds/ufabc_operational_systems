\chapter{Problema 2}

\subsection*{Código}


\begin{lstlisting}[style=CStyle]
    #include <sys/wait.h> /* system call - wait */
    #include <stdint.h> /* system call - wait */
    #include <stdlib.h> /* system call - exit */
    #include <unistd.h> /* system call - fork, exec, sleep */
    #include <stdio.h>
    
    /* Lib - System Call Signal */
    #include <signal.h>
    
    // Variaveis globais
    int file1Open = 1; /* Arquivo hipotetico 1 */
    int file2Open = 1; /* Arquivo hipotetico 2 */
    int valor1 = 500;
    
    void finish_process() {
        file1Open = 0;
        file2Open = 0;
        printf("Arquivos fechados!\n");
        printf("O conteúdo de 'valor1' é: %d\n", valor1);
        printf("Encerrando o processo...\n");
        exit(0);
        printf("Processo encerrado.\n");
    }
    
    /* Definicao da funcao em C que ira tratar das interrupcoes */
    void signal_handler(int signum) {
        if (signum == SIGTERM) {
            finish_process();
        }
        else if (signum == SIGINT) {
            char deseja_parar;
            printf("Deseja realmente parar o processo?\nSe sim, digite x.\n");
            scanf("&c", deseja_parar);
            if (deseja_parar == 'x' || deseja_parar == 'X') {
                finish_process();
            }
        } 
    }
    
    
    
    
    
    int main() {
        signal(SIGINT, signal_handler);
        signal(SIGTERM, signal_handler);
    
        pid_t pid;
        pid = fork();
        if (!pid) {
            printf( "Sou o FILHO. %d\n", getpid() );
            return 0;
        }
        else if (pid) {
            printf( "Sou o PAI. %d\n", getpid() );
            valor1 /= 20;
            wait(NULL);
            while (1);
            printf(
                "O status do file1Open eh %d e o file2Open eh %d \n", 
                file1Open, file2Open
            );
            printf(" O valor1 = %d\n",valor1);
            return 0;
        }
    }
    
\end{lstlisting}
